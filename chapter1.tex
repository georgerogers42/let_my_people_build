\startchapter[title={Let My People Build!}]
\startsection[title={Let My People Build!}]
Let my people build!
Free my people from the tyranny of Urban Growth Boundaries, Zoning, and Building Codes!
Let them build a city that is driven by the market, not the whims of rich political elites, at the expense of the poor!
The poor and middle class in this country are held down by excessive regulation at every level of politics from  City Hall all the way up to Washington DC.
This book will describe the ways City Hall harms the poor and middle class to benefit itself!
\stopsection
\startsection[title={Why is housing unaffordable?}]
Why is housing unaffordable?
\startitemize[A]
\item Developers
\item Rich Foreigners
\item Rich Newcomers
\item All of the above
\item Government, and the anti-growth misanthropy of the Green Movement.
\stopitemize
If you answered any of the items: A, B, C, or D; you are both wrong and part of the problem!
For if you think you have identified the problem---but incorrectly---you are the problem!

Let us go through the alleged culprits and show why they are not the problem but actually the solution!

\startsection[title={The Housing Lifecycle}]
A honest developer (which were all developers---before you screwed up the housing market! and are the developers you are against!) makes money by building housing.
Therefore demonizing developers for building housing for rich people makes the problem worse by making sure less housing is built! This in turn leads to more demonization, which leads to the doubling down on stupid!
This is in contrast to Texas and especially Houston, which are an honest developers paradise!
These markets have a rubber stamp permitting process which increases supply of housing leading to lower costs, and more honest developers!
\stopsection
\startsection[title={Filtering}]
Filtering is the process where luxury housing becomes affordable.
This occurs when the rich people move out to greener pastures and less affluent people take their place.

Let us assume for a second that we have a city with a static population. A developer builds a development with 1000 luxury condos, most people cannot afford to sit on a house/condo (especially with the second law of thermodynamics in play), therefore they sell their older house/condo, thereby putting it on the market.
This older home is less luxurious than the previous one so it most likely has a lower price (also because of the static population assumption more supply with equal demand equals lower prices).
This process is called {\sl Filtering}.

Filtering reduces prices in markets, especially ones that allow free building of housing such as Houston.
\stopsection
\stopchapter
